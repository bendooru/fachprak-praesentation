%
% THE BEER-WARE LICENSE (Rev. 42):
% Ronny Bergmann <bergmann@mathematik.uni-kl.de> wrote this file. As long as you
% retain this notice you can do whatever you want with this stuff. If we meet
% some day, and you think this stuff is worth it, you can buy me a beer or
% coffee in return.
%
% This file is just to get started - You need the corresponding Logo
%


%\documentclass[german,10pt,xcolor=colortbl,compress
%,draft
%]{beamer}
\documentclass[german,10pt,xcolor=colortbl,compress,handout]{beamer}
\usepackage[macce]{inputenc}
\usepackage[OT1]{fontenc}
\usepackage{calc}
\usepackage[ngerman]{babel} % Neue Rechtschreibung
\usepackage{amsmath,amsthm,amssymb,euscript} % AMS-LaTeX  
\usepackage{enumerate,graphicx}
\usepackage{listings}

% Load Theme
\usetheme[noptsans,navigation=true, FB=Mathematik, frametotal=true]{TUKL}
%
\setbeamertemplate{navigation symbols}{}
\title{Real root isolation}
\subtitle{Fachpraktikum}
\date[]{\today}
\author[]{Dominik Bendle, Clara Petroll}
\institute[]{AG Algebra, Geometrie und Computeralgebra \\ Betreuer: Janko B�hm}
%Setze ein Logo auf der Titelseite unten rechts
\renewcommand{\theSecondLogo}{}

\begin{document}
	\maketitle	
	\begin{frame}{Inhalt}
		\tableofcontents
	\end{frame}
	\section{Einleitung}
	\begin{frame}{Aufgabenstellung}
		\begin{itemize}
		\item Gegeben: nulldimensionales Radikalideal $J \subseteq \mathbb{R}[x_1, \hdots, x_n]$ und eine Box $B=I_1 \times \cdots \times I_n$.
		\pause
		\item Gesucht: Finde Boxen $B_i \subseteq B$, sodass jede Box $B_i$ genau ein Element aus $V(J)$ enth�lt.
		\end{itemize}
	\pause
	Warum macht diese Aufgabenstellung Sinn?
	\pause	
	\begin{lemma}
	Sei $J \subseteq \mathbb{R}[x_1, \hdots, x_n]$, dann sind �quivalent:
	\begin{itemize}
		\item[a)] $J$ ist nulldimensional.
		\item[b)] $V(J)$ ist endlich.
	\end{itemize}
	\end{lemma}
	
	\end {frame}
	
	\begin{frame}{Verfahren}
		\begin{itemize}
		\item Intervallarithmetik und Exklusion, um festzustellen dass in einer Box keine Nullstelle liegt
		\pause
		\item multivariater Interval Newton Step um Eindeutigkeit einer L�sung in einer Box fest zu stellen
		\pause 
		\item Implementierung in Singular-Library
		\end{itemize}	
	\end{frame}
	
	\section{Der Algorithmus}
	\begin{frame}{Intervallarithmetik}
		\begin{definition} 
		Sei $\textbf{R} \subseteq \mathbb{R}^2$ die Menge aller Intervalle $[a,b], a < b$. 
		Eine Intervall-Erweiterung einer Abbildung $f: \mathbb{R}^n \to \mathbb{R}$ ist eine Abbildung $\textbf{f}: \textbf{R}^n \to \textbf{R}$, sodass $f(x) \in \textbf{f}(\textbf{x}) $ f�r alle 			$x \in \mathbb{R}$.
		\end{definition}
		Damit: simpler Exklusionstest implementierbar. \bigskip
		
		\emph{Problem:} Wie bekommt man Eindeutigkeit einer Nullstelle in einer Box? 
	\end{frame}
	
	\begin{frame}{Interval Newton Step}
		\begin{itemize}
		\pause
		\item Idee: Wollen Testfunktion $T$, die zu gegebener Funktion $f: \mathbb{R}^n \to \mathbb{R}$ und einer Box $B_i$ aussagt:
		 	\begin{itemize}
			\item $T(f,B_i)=-1$, falls keine Nullstelle von f in der Box $B_i$ liegt
			\item $T(f,B_i)=0$, falls keine Aussage getroffen werden kann
			\item $T(f,B_i)=1$, falls genau eine Nullstelle von f in der Box $B_i$ liegt
			\end{itemize}
		\end{itemize}
		\pause
		Dazu berechne den Newton-Step von $f$ und einer Box $\textbf{x}$:
		$$ N(f, \textbf{x})= \hat{x} - \textbf{f'}(\textbf{x})^{-1}f(\hat{x})$$
		(Dabei $\hat{x} \in \textbf{x}$ beliebig, $\textbf{f'}$ Intervall-Erweiterung der Jacobi-Matrix von $f$)
	\end{frame}
	
	\begin{frame}{Interval Newton Step}{Anwendung auf unser Problem}
		\begin{theorem}
		Sei $I$ ein nulldimensionales Radikalideal des Polynomrings $K[x_1, \hdots, x_n]$, wobei $K$ ein K�rper ist. Dann kann I von $n$ Elementen erzeugt werden. 
		\end{theorem}
		
		\begin{itemize}
		\item nulldimensionales Radikalideal $I \subseteq \mathbb{R}[x_1, \hdots, x_n]$ somit gegeben als eine Menge an Erzeugern $\{f_1, \hdots f_n\}, f_i \in \mathbb{R}[x_1, \hdots, x_n], 			i=1,\hdots,n$
		\pause
		\item betrachte nun die Funktion $f:=\left( \begin{array}{c} f_1\\ \vdots \\ f_n \end{array} \right)$, $f:\mathbb{R}^n\to \mathbb{R}$  \\
			
		%\item Das haben wir weil... (computational methods in commutative Algebra and Algebraic geometry -> let I zeroddim radical then I generated by n elements)
		%\pause
		\end{itemize}
		$\implies$ K�nnen den Interval Newton Step auf unser Ideal anwenden.
	\end{frame}
	
	\begin{frame}{Interval Newton step}
		\begin{theorem}
		Sei $\textbf{x}$ eine Box im $\mathbb{R}^n$ und $f:\mathbb{R}^n\to \mathbb{R}$. Dann gilt:
		\begin{itemize}
		\item[a)] Jede L�sung in $\textbf{x}$ liegt auch in $N(f,\textbf{x})$. \\
		\pause
		\item[b)] Gilt  $N(f,\textbf{x}) \subseteq int(\textbf{x})$, dann existiert genau eine L�sung in $\textbf{x}$, d.h. $T(f,\textbf{x})=1$.
		\end{itemize}
		\end{theorem}
		\pause
		
		\textbf{Implementierung:} 
		\\
		Bisektion bis f�r jede der entstehenden Boxen eine der folgenden Bedingungen erf�llt ist:
		\pause
		\begin{itemize}
		\item In der Box liegt genau eine L�sung.
		\item \glqq Gr��e\grqq{} der Box unter einer gewissen Schranke.
		\end{itemize}
		
	\end{frame}
	
	\begin {frame}[fragile]{testPolyBox}
	\lstset{language=C}
	\begin{lstlisting}
	for (i = 1; i <= ncols(I); i++) {
		   tmp = evalPolyAtBox(I[i], B);
		   // check if 0 contained in every interval
		   // return -1 if not
		   if (tmp[1]*tmp[2] > 0){return(-1, B); } 
		}
	 
	//now: Newton Step
	box Bcenter = boxCenter(B);
	ivmat J = evalJacobianAtBox(I, B);
	list inverse = ivmatGaussian(J);

	// only continue if J is invertible
	if (!inverse[1]) { return(0, B); }
	ivmat Jinverse = inverse[2];

	\end{lstlisting}
	\end{frame}
	\begin {frame}[fragile]{testPolyBox}
	\lstset{language=C}
	\begin{lstlisting}
	
	// calculate Bcenter - f(B)^(-1)f(Bcenter)	
	box fB = evalIdealAtBox(I, Bcenter);  
	fB = Bcenter - (Jinverse * fB);
	
	// algorithm will not process box further, so do not modify
	int laststep = boxIsInterior(fB, B);
	
	// else intersection is empty or non-trivial
	def Bint = intersect(B, fB);
	
	\end{lstlisting}
	\end{frame}
	\begin {frame}[fragile]{testPolyBox}
	\lstset{language=C}
	\begin{lstlisting}

	// if intersection is empty Bint == -1
	if (typeof(Bint) == "int") { return(-1, B); }

	// attempt simplification of fractions
	... 
	...
		
	if (laststep) { return(1, B); }

	// no condition could be verified
	return(0, B);
    
	\end{lstlisting}

	\end{frame}
	
	\section{Probleme und Verbesserungen}
	\begin{frame}{Schlechte Laufzeit}
		\begin{itemize}
		\item Intervallarithmetik als \glqq newstruct\grqq{} in Singular implementiert
		\item Invertieren einer Intervall-Matrix NP-schwer
		\end{itemize}
	\bigskip
	\textbf{Unsere Verbesserungen:}
	\pause
		\begin{itemize}
		\item Auslagerung der Intervallarithmetik in C++  (dynamische Moduln in Singular)
		\item Einbinden dieses Teils des Codes als .so 
		\end{itemize}
	\end{frame}
	
	\begin{frame}{Gro�e Br�che}
		\begin{itemize}
			\pause
			\item Schneiden von Boxen $\rightarrow$ teilweise sehr gro�e Br�che $\rightarrow$ langsam
			\pause
			\item Idee: "Runden" von Br�chen 
			\pause
			\item Umsetzung: Vergr��ere die Box ein bisschen, um sch�nere Darstellung zu erzielen
		\end{itemize}
		\bigskip
		\pause
		
		$\implies$ Dadurch annehmbare Laufzeit sogar bei f�nf Variablen.
	\end{frame}
	
	\begin{frame}{Schlechte Startboxen}
	\pause
	\begin{lemma}
	Sei $I$ ein Ideal des Polynomrings $K[x_1, \hdots, x_n]$, wobei $K$ ein K�rper ist. Die folgenden Aussagen sind �quivalent:
	\begin{itemize}
	 	\item[a)] $I$ ist nulldimensional.
		\item[b)] $I \cap K[x_i] \neq \{0\}$ f�r $i=1,\hdots, n$.
	\end{itemize}
	\pause
	\end{lemma}
		\begin{itemize}
		\item Bestimme Gr�bnerbasis des Ideals mit einer Eliminationsordnung.
		\pause
		\item $\implies$ Erste Gleichung h�ngt nur von einer Variablen ab.
		\pause
		\item Wende darauf unseren Exclusiontest an $\rightarrow$ bekommen bessere Startboxen f�r eine Variable.
		\end{itemize}
	\end{frame}
	
	\begin{frame}{Boxen mit eindeutigen L�sungen}
		\begin {itemize}
		%\item Wollen nur Boxen mit eindeutigen L�sungen.
		\item Bisher: Eingabe bestehend aus Ideal, Startbox, Schranke f�r die Gr��e der Boxen. 
		\\ $\rightsquigarrow$ \emph{Ziel:} Wollen keine Boxen, die auf Grund ihrer Gr��e vom Algorithmus nach Terminierung ausgegeben werden.
		\end{itemize}
	\pause
	\textbf{Unsere Verbesserungen:}
	\begin {itemize}
	\pause
	\item Falls Nullstelle auf der Ebene liegt, an der wir die Box bei Bisektion zerteilen $\implies$ nie Eindeutigkeit der L�sung mit Newton Step.
	\pause
	\item Verbesserung der Algorithmus \glqq splitBox \grqq{} durch:
	\pause
		\begin{itemize}
		\item Werte Ideal an der Zerteilungsebene aus. 
		\item Gr�bnerbasistest
		\end{itemize}
	\end{itemize}
	
	\end{frame}
	% warum k�nnen wir newton step anwenden?
	% nulldimensionales Radikalideal -> immer n erzeuge??, auf jeden fall finite vanishing locus
	
	% frame zu Unser Vorgehen 
	%Probleme: sehr langsam-> L�sung?
	%Codeausschnitte
	
	%Literatur
	
	\begin{frame}{Literaturverzeichnis}
	\bibliographystyle{alphadin}
	\begin{thebibliography}{9}
	\bibitem{Moore} Cloud, Kearfott, Moore
	\textit{Introduction to Interval Analysis}.
	Society for Industrial and Applied Mathematics, 2009.
	
	
	\bibitem{Vasconcelos} Eisenbud, Grayson, Herzog, Stillman, Vasconcelos
	\textit{Computational Methods in Commutative Algebra and Algebraic Geometry}. Springer Verlag Berlin-Heidelberg, 3. Auflage 2004.
	
	\bibitem[A]{Sommese} Andrew J. Sommese and Charles W. Wampler
	\textit{The Numerical Solution of Systems of Polynomials}. [\textit{Arising in Engineering and Science}].
	World Scientific Publishing Co. Pte. Ltd., 2005.
	
	\end{thebibliography}
	
	\end{frame}
	
\end{document}